\documentclass[12pt, a4paper]{article}
\usepackage[utf8]{inputenc}
\usepackage[brazilian]{babel} % Hifenização e dicionário
\usepackage[left=3.00cm, right=2.00cm, top=3.00cm, bottom=2.00cm]{geometry}
\usepackage{enumitem} % Para itemsep etc
\usepackage{longtable} % Dependência do longtabu
\usepackage{tabu} % Para melhor criação de tabelas
\usepackage{listings} % Para códigos
\usepackage{lstautogobble} % Códigos indentados corretamente
\usepackage{color} % Para coloração de códigos
\usepackage{zi4} % Para fonte de códigos
\usepackage{parskip} % Linha em branco entre parágrafos em vez de recuo
\usepackage{graphicx}
\usepackage{float}
\usepackage{verbatim}
\usepackage[autostyle]{csquotes}
\usepackage[breaklinks]{hyperref}

\usepackage{listings}
\lstset{
    autogobble,
    columns=fullflexible,
    showspaces=false,
    keepspaces=true,
    showtabs=true,
    breaklines=true,
    showstringspaces=false,
    breakatwhitespace=true,
    escapeinside={(*@}{@*)},
    commentstyle=\color{greencomments},
    keywordstyle=\color{bluekeywords},
    stringstyle=\color{redstrings},
    numberstyle=\color{graynumbers},
    basicstyle=\ttfamily\footnotesize,
    frame=l,
    framesep=12pt,
    xleftmargin=12pt,
    tabsize=4,
    captionpos=b,
}

\newcommand{\ic}[1]{\textbf{\lstinline{#1}}}

\DeclareGraphicsExtensions{.pdf}

\begin{document}

\begin{center}
    \textsc{Universidade Federal do Rio Grande do Norte} \\
    \textsc{Departamento de Informática e Matemática Aplicada}
\end{center}

\bigskip

\begin{tabular}{@{}ll@{}}
    \emph{Disciplina:} & DIM0406 --- Algoritmos Avançados \\
    \emph{Docente:}    & Sílvia Maria Diniz Monteiro Maia \\
    \emph{Discente:}   & Felipe Cortez de Sá \\
\end{tabular}

\bigskip

\begin{center}
\large \textbf{GRASP-VNS aplicado ao problema de Steiner com rotulação mínima}
\end{center}

\section{Introdução}
Neste relatório é apresentado um algoritmo metaheurístico para resolver o
problema da árvore de Steiner com rotulação mínima. O algoritmo é uma
combinação do \emph{Greedy Randomized Adaptative Search Procedure} apresentado
por Consoli et al \cite{consoli} com sua fase de busca local substituída por
uma \emph{Variable Neighbourhood Search}. Os resultados são comparados com o
algoritmo exato implementado na primeira unidade, é feita uma análise de
complexidade, é descrito como são gerados os casos de teste, uma tabela mostra
os resultados obtidos para uma média de dez execuções de caso de teste. Além
disso, é feito um adendo para o relatório do primeiro trabalho, apresentando os
resultados previamente faltantes a conclusão.

\section{Metaheurísticas utilizadas}
\subsection{GRASP}
O \emph{Greedy Randomized Adaptative Search Procedure} é comumente utilizado em
problemas de otimização combinatória. A cada iteração, é realizada uma fase de
construção, em que se gera uma solução para o problema e posteriormente uma
fase de busca local, procurando um mínimo local na vizinhança da solução
gerada. Se a melhor solução global é encontrada na iteração, atualiza-se a
variável que contém a melhor solução. As duas fases são repetidas até o
critério de parada ser satisfeito, podendo ser o número de iterações ou o tempo
de execução, por exemplo. Na fase de construção, é criada uma lista de
candidatos restritos,
função gulosa, restricted candidate list possui os candidatos cujos elementos
adicionados minimizam os custos incrementais. O elemento é selecionado
aleatoriamente da RCL.
inicialização múltipla.

\subsection{VNS}
O \emph{Variable Neighbourhood Search} faz uso de múltiplas estruturas de
vizinhança, explorando vizinhanças cada vez mais distantes e maiores assim que
o ótimo local é obtido por uma dessas vizinhanças. 

\section{Metaheurística aplicada ao problema}

\section{Complexidade}

\section{Casos teste utilizados}
Os casos teste utilizados são gerados automaticamente por um programa separado
de acordo com parâmetros de entrada. Os parâmetros são \ic{SIZE}, a quantidade
de nós do grafo, \ic{COLORS}, o número de rótulos, \ic{DENSITY}, a proporção de
arestas para cada nó, e \ic{BASIC}, a quantidade de nós básicos. \ic{DENSITY}
funciona percorrendo a matriz de adjacência que representa o grafo e de acordo
com a probabilidade definida (sendo 0 e 100 equivalentes a 0\% e 100\%,
respectivamente) adicionando ou não uma aresta de rotulação aleatória ligando
dois nós. O arquivo gerado é então passado para o programa principal.

\section{Resultados}

\section{Correções do primeiro trabalho}

\begin{thebibliography}{9}

\bibitem{consoli}
S. Consoli, K. Darby-Dowman, N. Mladenovic, J.A. Moreno-Perez.
\textit{Variable neighbourhood search for the minimum labelling Steiner tree
problem}.
Annals of Operations Research, 2009.
\tiny
\\\texttt{\url{https://www.researchgate.net/publication/225327721_Variable_neighbourhood_search_for_the_minimum_labelling_Steiner_tree_problem}}
\normalsize

\bibitem{graphviz}
\textit{Graphviz --- Graph Visualization Software}. \\
\url{http://www.graphviz.org/}

\bibitem{handbook}
Glover, F., Kochenberger, G. A. et al.
\textit{Handbook of Metaheuristics}.
Kluwer Academic Publishers
\tiny
\normalsize

\bibitem{executiontime}
\textit{Stack Overflow --- Execution time of a C program}. \\
\url{http://stackoverflow.com/questions/5248915/execution-time-of-c-program}

\end{thebibliography}

\end{document}
