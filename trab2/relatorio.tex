\documentclass[12pt, a4paper]{article}
\usepackage[utf8]{inputenc}
\usepackage[brazilian]{babel} % Hifenização e dicionário
\usepackage[left=3.00cm, right=2.00cm, top=3.00cm, bottom=2.00cm]{geometry}
\usepackage{enumitem} % Para itemsep etc
\usepackage{longtable} % Dependência do longtabu
\usepackage{tabu} % Para melhor criação de tabelas
\usepackage{listings} % Para códigos
\usepackage{lstautogobble} % Códigos indentados corretamente
\usepackage{color} % Para coloração de códigos
\usepackage{zi4} % Para fonte de códigos
\usepackage{parskip} % Linha em branco entre parágrafos em vez de recuo
\usepackage{graphicx}
\usepackage{float}
\usepackage{verbatim}
\usepackage[autostyle]{csquotes}
\usepackage[breaklinks]{hyperref}

\usepackage{listings}
\lstset{
    autogobble,
    columns=fullflexible,
    showspaces=false,
    keepspaces=true,
    showtabs=true,
    breaklines=true,
    showstringspaces=false,
    breakatwhitespace=true,
    escapeinside={(*@}{@*)},
    commentstyle=\color{greencomments},
    keywordstyle=\color{bluekeywords},
    stringstyle=\color{redstrings},
    numberstyle=\color{graynumbers},
    basicstyle=\ttfamily\footnotesize,
    frame=single,
    framesep=12pt,
    xleftmargin=12pt,
    tabsize=4,
    captionpos=b,
}

\newcommand{\ic}[1]{\textbf{\lstinline{#1}}}

\DeclareGraphicsExtensions{.pdf}

\begin{document}

\begin{center}
    \textsc{Universidade Federal do Rio Grande do Norte} \\
    \textsc{Departamento de Informática e Matemática Aplicada}
\end{center}

\bigskip

\begin{tabular}{@{}ll@{}}
    \emph{Disciplina:} & DIM0406 --- Algoritmos Avançados \\
    \emph{Docente:}    & Sílvia Maria Diniz Monteiro Maia \\
    \emph{Discente:}   & Felipe Cortez de Sá \\
\end{tabular}

\bigskip

\begin{center}
\large \textbf{GRASP-VNS aplicado ao problema de Steiner com rotulação mínima}
\end{center}

\section{Introdução}
Neste relatório são apresentados os algoritmos \emph{Greedy Randomized
Adaptative Search Procedure} e \emph{Variable Neighbourhood Search} para
solução do problema da árvore de Steiner com rotulação mínima. Os princípios de
cada técnica são descritos, assim como a maneira em que foram utilizados para
resolver especificamente o problema. É realizada uma análise de complexidade em
tempo para as implementações, identificados os gargalos. Explica-se como foram
gerados os casos de teste e os seus parâmetros e são apresentados resultados
comparando o resultado das execuções para cada técnica, incluindo o algoritmo
exato desenvolvido na segunda unidade. EXPERIMENTOS COMPUTACIONAIS PARA
DEFINIÇÃO DO ALGORITMO! Além disso, é feito um adendo para o relatório do
primeiro trabalho, apresentando os resultados e conclusão previamente
faltantes.

\section{Metaheurísticas utilizadas}
\subsection{GRASP}
O \emph{Greedy Randomized Adaptative Search Procedure} é comumente utilizado em
problemas de otimização combinatória. A cada iteração, é realizada uma fase de
construção, em que se gera uma solução para o problema e posteriormente uma
fase de busca local, procurando um mínimo local na vizinhança da solução
gerada. Se a melhor solução global é encontrada na iteração, atualiza-se a
variável que contém a melhor solução. É um algoritmo de inicialização múltipla,
ou seja, as duas fases são repetidas até o critério de parada ser satisfeito,
podendo ser o número de iterações ou o tempo de execução, por exemplo. Na fase
de construção, é criada uma lista de candidatos restritos, possuindo os
candidatos cujos elementos adicionados minimizam os custos incrementais. O
elemento é selecionado aleatoriamente da RCL.

\begin{lstlisting}[caption=Pseudocódigo para GRASP, basicstyle=\ttfamily\scriptsize]
grasp(limite) {
    col* = {}
    int no_improv = 0

    while(no_improv < limite) {
        col = {}
        construct(col)
        local(col) ou vns(col)
        if(card(col) < card(col*)) {
            col* = col
            no_improv = 0
        } else {
            ++no_improv
        }
    }
}

construct() {
    if(iteration > 2) {
        rcl = {1, 1, ... , 1}
        col[random()] = 1
    } else {
        rcl = {0, 0, ... , 0}
    }

    while(comp(c) > 1) {
        rcl = argmin comp(c)
        col = col (*@$\cup$@*) rcl[random()]
    }
}
\end{lstlisting}

\subsection{VNS}
O \emph{Variable Neighbourhood Search} faz uso de múltiplas estruturas de
vizinhança, explorando comumente espaços cada vez mais distantes e maiores,
portanto mais custosos. Para fugir de mínimos locais, o algoritmo possui uma fase
de agitação, em que a solução encontrada pode ser trocada por uma pior a fim de
diversificar a busca, explotando melhor o espaço.

\begin{lstlisting}[caption=Pseudocódigo para VNS, basicstyle=\ttfamily\scriptsize]
vns(col, kmax)
col2 = col
k = 1
while(k <= kmax) {
    shaking(col2, k)
    local(col2)
    if(card(col2) < card(col)) {
        col = col2
        k = 1
    } else {
        ++k
    }
}

shaking(col, k) {
    col2 = col
    for(i in 1..k) {
        if(i <= card(col)) {
            col[random(cores utilizadas em col)] = 0
        } else {
            col[random(cores nao utilizadas em col)] = 1
        }
    }

    while(comp(c) > 1) {
        melhores = argmin comp(c)
        col2[random(melhores)] = 1
    }
}
\end{lstlisting}


\section{Metaheurística aplicada ao problema}
\subsection{GRASP}

\section{Complexidade}

\section{Casos teste utilizados}
Os casos teste utilizados são gerados automaticamente por um programa
\ic{generate.c} de acordo com parâmetros de entrada. Os parâmetros são
\ic{SIZE}, a quantidade de nós do grafo, \ic{COLORS}, o número de rótulos,
\ic{DENSITY}, a proporção de arestas para cada nó, e \ic{BASIC}, a quantidade
de nós básicos. \ic{DENSITY} funciona percorrendo a matriz de adjacência que
representa o grafo e de acordo com a probabilidade definida (sendo 0 e 100
equivalentes a 0\% e 100\%, respectivamente) adicionando ou não uma aresta de
rotulação aleatória ligando dois nós. O arquivo gerado é então passado para o
programa principal.

A fim de comparar os resultados com o trabalho realizado por Consoli
\cite{consoli}, os parâmetros dos testes são os mesmos, isto é, tem-se uma
combinação entre \ic{DENSITY} $ \in \{80, 50, 20\} $, e para \ic{SIZE} $ = 100 $,
\ic{COLORS} $ \in \{25, 50, 100, 125\} $, para \ic{SIZE} $ = 500 $, \ic{COLORS}
$ \in \{125, 250, 500, 625\} $. \ic{BASIC} recebe $ 20\% $ de \ic{SIZE}, isto é,
20 e 100, respectivamente.

O código que gera os arquivos de caso teste para os parâmetros desejados está
em \ic{generate.py}.

\section{Resultados}

\section{Correções do primeiro trabalho}
\subsection{Técnica utilizada}
\subsection{Resultados}
\subsection{Conclusão}

\begin{thebibliography}{9}

\bibitem{consoli}
S. Consoli, K. Darby-Dowman, N. Mladenovic, J.A. Moreno-Perez.
\textit{Variable neighbourhood search for the minimum labelling Steiner tree
problem}.
Annals of Operations Research, 2009.
\tiny
\\\texttt{\url{https://www.researchgate.net/publication/225327721_Variable_neighbourhood_search_for_the_minimum_labelling_Steiner_tree_problem}}
\normalsize

\bibitem{graphviz}
\textit{Graphviz --- Graph Visualization Software}. \\
\url{http://www.graphviz.org/}

\bibitem{handbook}
Glover, F., Kochenberger, G. A. et al.
\textit{Handbook of Metaheuristics}.
Kluwer Academic Publishers
\tiny
\normalsize

\bibitem{executiontime}
\textit{Stack Overflow --- Execution time of a C program}. \\
\url{http://stackoverflow.com/questions/5248915/execution-time-of-c-program}

\end{thebibliography}

\end{document}
