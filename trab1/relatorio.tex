\documentclass[12pt, a4paper]{article}
\usepackage[utf8]{inputenc}
\usepackage[brazilian]{babel} % Hifenização e dicionário
\usepackage[left=3.00cm, right=2.00cm, top=3.00cm, bottom=2.00cm]{geometry}
\usepackage{enumitem} % Para itemsep etc
\usepackage{longtable} % Dependência do longtabu
\usepackage{tabu} % Para melhor criação de tabelas
\usepackage{listings} % Para códigos
\usepackage{lstautogobble} % Códigos indentados corretamente
\usepackage{color} % Para coloração de códigos
\usepackage{zi4} % Para fonte de códigos
\usepackage{parskip} % Linha em branco entre parágrafos em vez de recuo
\usepackage{graphicx}
\usepackage{float}
\usepackage{verbatim}
\usepackage[autostyle]{csquotes}
\usepackage[breaklinks]{hyperref}

\usepackage{listings}
\lstset{
    autogobble,
    columns=fullflexible,
    showspaces=false,
    keepspaces=true,
    showtabs=true,
    breaklines=true,
    showstringspaces=false,
    breakatwhitespace=true,
    escapeinside={(*@}{@*)},
    commentstyle=\color{greencomments},
    keywordstyle=\color{bluekeywords},
    stringstyle=\color{redstrings},
    numberstyle=\color{graynumbers},
    basicstyle=\ttfamily\footnotesize,
    frame=l,
    framesep=12pt,
    xleftmargin=12pt,
    tabsize=4,
    captionpos=b,
}

\newcommand{\ic}[1]{\textbf{\lstinline{#1}}}

\DeclareGraphicsExtensions{.pdf}

\begin{document}

\begin{center}
    \textsc{Universidade Federal do Rio Grande do Norte} \\
    \textsc{Departamento de Informática e Matemática Aplicada}
\end{center}

\bigskip

\begin{tabular}{@{}ll@{}}
    \emph{Disciplina:} & DIM0406 --- Algoritmos Avançados \\
    \emph{Docente:}    & Sílvia Maria Diniz Monteiro Maia \\
    \emph{Discente:}   & Felipe Cortez de Sá \\
\end{tabular}

\bigskip

\begin{center}
\large \textbf{Algoritmo exato para problema de Steiner com rotulação mínima}
\end{center}

\section{Introdução}
Neste relatório é apresentado o problema da árvore de Steiner com rotulação
mínima e um algoritmo exato para resolvê-lo baseado na técnica do
\emph{branch and bound}, sugerida por Consoli et al \cite{consoli}.

\section{Problema}
O problema de Steiner com rotulação mínima é a junção de dois problemas
similares: \emph{Minimum Labelling Spanning Tree}, que busca uma árvore
geradora para um grafo $ G = (V, E, L) $ utilizando a menor quantidade de
rótulos possível, e o problema da árvore de Steiner, que busca num grafo
ponderado $ G = (V, E, w) $ uma árvore que conecta determinados vértices
básicos $ Q \subseteq V $ minimizando o custo das arestas. A combinação desses
problemas consiste em achar uma árvore que contenha todos os vértices básicos $
Q $ utilizando o menor número de rótulos possível.

Formalmente, dado um grafo $ G = (V, E, L) $, sendo $ V $ o conjunto de
vértices, $ E $ o conjunto de arestas, $ L $ o conjunto de rótulos para as
arestas e $ Q \subseteq V $ um conjunto de vértices básicos, uma árvore de
Steiner com rotulação mínima contém todos os vértices básicos conectados
possivelmente utilizando os vértices não básicos $ V - Q $ minimizando o número
de rótulos utilizados.


O algoritmo tem aplicações na área de construção de circuitos \emph{VLSI}, em
que se deseja minimizar o cabeamento utilizado para conectar pinos,
telecomunicações, engenharia civil, entre outros \cite{robins}.

\begin{figure}[H]
    \centering
    \includegraphics[width=0.50\textwidth]{graph_1}
    \caption{Um exemplo do problema com 10 vértices, 3 deles básicos e 3 cores}
    \label{fig:graph_1}
\end{figure}

\section{Algoritmo e técnica de solução}
O algoritmo foi implementado utilizando \emph{C} em um único arquivo
\ic{steiner.c} que não recebe entrada (já que seus casos de teste são gerados
automaticamente) e que gera como saída dois arquivos \ic{.dot}, que podem ser
transformados em uma visualização do grafo inicial e final através do programa
\ic{graphviz} \cite{graphviz}.

O funcionamento é baseado no algoritmo exato descrito por Consoli et al, que
adiciona a técnica de \emph{branch and bound} ao algoritmo guloso proposto por
Cerulli et al. \cite{cerulli}

%if (*@ $ \lvert C \rvert  < \lvert C^* \rvert $ @*) {
\begin{lstlisting}[caption=Pseudocódigo]
(*@$C = \{\}$@*)
(*@$H = (V, E(C)), E(C) = \{e \in E : L(e) \in C\}$@*)
(*@$C^* = L $@*)
(*@$H^* = (V, E(C^*)), E(C^*) = \{e \in E : L(e) \in C\}$@*)
(*@$Comp(C) $@*) // componentes conexos de (*@ (Q, E(C)) @*)

Test(C) {
    if ((*@$ |C| < |C^*| $@*)) {
        if ((*@$ Comp(C) = 1 $@*)) {
            (*@$ C^* \leftarrow C $@*)
        } else if ((*@$ |C| < |C^*| - 1 $@*)) {
            for each ((*@$ c \in (L - C) $@*)) {
                Test((*@$ C \cup \{ c \} $@*))
            }
        }
    }
}
\end{lstlisting}

O algoritmo adiciona uma nova cor a cada chamada do procedimento \ic{Test},
permitindo a exploração do espaço de busca. $ C^* $ guarda a melhor solução até
o momento numa variável global e $ |C^*| $ informa quantos rótulos essa melhor
solução possui. A poda (ou \emph{bound}) é feita pela comparação $ |C| < |C^*|
- 1 $, que ao ser avaliada como falsa significa que ao adicionar uma nova cor a
$ |C| $, sua cardinalidade será igual à da melhor solução, isto é, não se terá uma
solução melhor e portanto não adianta explorar mais soluções a partir do
conjunto $ C $.

\ic{Comp(C)} é calculado através de uma busca em profundidade.

Após executar \ic{Test}, tem-se $ C^* $ atualizado com a melhor combinação de
rótulos que garantem apenas um componente conexo contendo todos os vértices
básicos $ Q $.

\begin{figure}[H]
    \centering
    \includegraphics[width=0.50\textwidth]{graph_2}
    \caption{Grafo com arestas em $ C^* $}
    \label{fig:graph_2}
\end{figure}

Para encontrar uma solução, basta achar uma árvore geradora desse grafo e
remover arestas não básicas $ V - Q $ com grau 1.

\section{Casos teste}
Os casos de teste são gerados automaticamente de acordo com quatro parâmetros
\begin{itemize}
\item \ic{SIZE}, a quantidade de vértices do grafo
\item \ic{COLORS}, a quantidade de rótulos (ou cores) em $ L $
\item \ic{DENSITY}, valor de 0 a 100 que define a quantidade de arestas
\item \ic{BASIC}, a quantidade de vértices básicos
\end{itemize}

\section{Resultados}
Não foi possível testar extensamente e comparar a eficiência do algoritmo para
casos grandes.

\begin{thebibliography}{9}

\bibitem{consoli}
S. Consoli, K. Darby-Dowman, N. Mladenovic, J.A. Moreno-Perez.
\textit{Variable neighbourhood search for the minimum labelling Steiner tree
problem}.
Annals of Operations Research, 2009.
\tiny
\\\texttt{\url{https://www.researchgate.net/publication/225327721_Variable_neighbourhood_search_for_the_minimum_labelling_Steiner_tree_problem}}
\normalsize

\bibitem{robins}
G. Robins, A. Zelikovsky.
\textit{Minimum Steiner Tree Construction}.
\tiny
\\\texttt{\url{http://www.cs.virginia.edu/~robins/papers/Steiner_chapter.pdf}}
\normalsize

\bibitem{graphviz}
\textit{Graphviz --- Graph Visualization Software}. \\
\url{http://www.graphviz.org/}

\bibitem{cerulli}
R. Cerulli, A. Fink, M. Gentili e S. Voß.
\textit{Extensions of the minimum labelling spanning tree problem}.
Journal of Telecommunications and Information Technology, 2006.
\tiny
\\\texttt{\url{https://www.researchgate.net/publication/228668519_Extensions_of_the_minimum_labelling_spanning_tree_problem}}
\normalsize
\end{thebibliography}

\end{document}
